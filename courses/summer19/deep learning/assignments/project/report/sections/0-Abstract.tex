\documentclass[../main.tex]{subfiles}
 
\begin{document}
\begin{abstract}

% motivate the problem
% details of the problem domain

Today's vehicle manufacturers do not tend to publish proprietary \ac{can} packet formats. This is a form of \textit{security through obscurity}---it makes reverse engineering efforts more difficult for would-be intruders---but obfuscating the \ac{can} data in this way does not adequately hide the vehicle's unique signature. Specifically, modern methods can effectively identify a vehicle's signature in a segment of its \ac{can} data, even if this data is unprocessed or limited in scope. In deep learning, this is a multiclass classification problem which asks the following question: given a sample of \ac{can} data, can we successfully determine which vehicle generated the sample?

% overview of dataset

\medskip
\noindent This research employs two datasets, one from \ac{ornl} and one from Stone et al (2018). \ac{ornl}'s corpus is comprised of nearly 2.5 gigabytes of data captured on the \ac{can} buses of nine different vehicles; Stone et al's corpus contains over 230 megabytes of data from 11 different vehicles. In this research, 1,024 bytes of sequential \ac{can} data constitute one data sample, so formatting and partitioning the datasets gives a new dataset of nearly three hundred thousand individual samples. We label every sample with its generating vehicle to enable fully supervised learning.

% overview of deep learning task
% implications/benefit of research

\medskip
\noindent We train two distinct deep learning models on this dataset. The results indicate that a standard \ac{mlp} can effectively classify these \ac{can} data samples. The results also indicate that a deep \ac{cnn} can classify the samples at a greater performance level than can the \ac{mlp}, but both models still surpass a balanced classification accuracy of 80\% on the full dataset. Clearly, one can effectively determine which vehicle generated a given sample of \ac{can} data. This erodes consumer safety: a sophisticated attacker who establishes a presence on an unknown vehicle can use similar techniques to identify the vehicle and better format attacks.

\end{abstract}
\end{document}
