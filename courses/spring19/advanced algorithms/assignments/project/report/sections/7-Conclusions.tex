\documentclass[../main.tex]{subfiles}
 
\begin{document}

In this research, we devised three different approaches to solving \prob, a multi-domain optimization problem. In doing so, we achieved multiple educational objectives: an understanding of \ac{npc} problem domains; an ability to describe and use various deterministic, nature-inspired, global, and local search algorithms with and without heuristics; and an ability to refine general search algorithms to solve specific problems (among many other objectives met). For this reason, this research is a resounding success.

It's somewhat of a success in other ways, too. Results, as described in Section \ref{sec:experiment}, indicate that an algorithmic approach to \probs can give feasible solutions for small-to-medium problem instances. Although the algorithms developed in Sections \ref{sec:deterministic}, \ref{sec:stochastic}, and \ref{sec:local} are not capable as-is of solving practical \probs instances, it's clear that, with powerful hardware and robust, well-developed heuristics, the \ac{usaf} may find strong results using one of these methods. Additionally, the experiment conducted is simple enough to allow for easy communication to a layman but complex enough to illustrate the partial success of A$^*$, \ac{sbs}, and \ac{lbs} for this problem domain. Some experiments in the literature are certainly more involved than ours, but many are not -- for this reason, we assert that ours is successful.

Because we closely followed and documented the \ac{pdad} design process, future researchers can easily reproduce the research presented in this report and utilize the algorithms described to solve their own, similar problems. Additionally, the widespread use of Python ensures that our code will remain relevant for many years to come. Clearly, both the algorithms employed -- A$^*$, \ac{sbs}, and \ac{lbs} -- and the code developed are of a high quality that may prove useful to future students and researchers.

\end{document}